%% Include table macros
%

\def\title#1{{\noindent \bf #1}}
\font\eightrm=cmr8
\let\sc=\eightrm % small caps (NOT a caps-and-small-caps font)
\def\ll#1{\leftline{\bf#1}}
\def\ttitle#1{\centerline{\bf #1}\bigskip\nobreak}
\def\bs{\bigskip}
\def\ms{\medskip}
\def\ss{\smallskip}
\def\mb{$\bullet$\ }

\input myindexing
\input twelvepoint
% \input moredefs

\twelvepoint


\title{7.6 Lab}

\bs

It’s back to the Pi app for this lab. The Docker image kiamol/ch05-pi can actually be used in different ways, and to run it as a web app, you need to override the startup command in the container spec. We’ve done that in the YAML files in previous chapters, but now we’ve been asked to use a standard approach to setting up the pod. Here are the requirements and some hints:

\ms

\mb     The app container needs to use a standard startup command that all Pods in our platform are using. It should run {\tt /init/startup}.

\mb   The Pod should use port 80 for the app container.

\mb     The Pod should also publish port 8080 for an HTTP server, which returns the version number of the app,

\mb The app container image doesn’t contain a startup script, so you’ll need to use something that can create that script and make it executable for the app container to run.

\mb  The app doesn’t publish a version API on port 8080 (or anywhere else), so you’ll need something that can provide that (it can just be any static text).”

\ms

\noindent The starting point is the YAML in ch07/lab/pi, which is broken
at the moment. You’ll need to do some investigation into how the
app ran in previous chapters and apply the techniques we’ve learned
in this chapter. You have plenty of ways to approach this one, and
you’ll find my sample solution in the usual place:

$$
https://github.com/sixeyed/kiamol/blob/master/ch07/lab/README.md.
$$


\bye

Excerpt From
Learn Kubernetes in a Month of Lunches
Elton Stoneman
https://itunes.apple.com/WebObjects/MZStore.woa/wa/viewBook?id=0
This material may be protected by copyright.
